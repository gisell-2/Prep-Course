#Bucles: Es una secuencia que repite varias veces un trozo de código, hasta que la condición de este bucle
deja de cumplirse. Es decir nos sirven para facilitar una tarea repetitiva. 
#bucle for: Se usan para realizar varias veces una tarea, se puede colocar la cantidad de veces que se 
repetirá. Además permite anidar statement, con if. 
Esté bucle se escribe de la siguiente manera:
palabra for(variable; condición de frenado; como incrementa el valor) 
               for(var i=0; i<=100; i++)
               // bloque de código
-for: Este bucle repite varias veces algo
 -variable: determina que es lo que cambiará, cuenta la cantidad de veces que cambiará
 -condición de frenado: determina cuando se detendrá de repetir el código
 -i++: significa que crezca de uno en uno
#bucle while: Es un statement, trabaja con boolean. Si la condición es verdadera el código se repetirá 
indefinidamente.

